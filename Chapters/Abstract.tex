\begin{center}
    \Large{\textbf{ABSTRACT}}\\
\end{center}
\vspace{1cm}
Satellite quantum key distribution is proposed to be an alternative method to establish intercontinental secure communication links instead of using optical fibre which is incapable of transmitting information in long distance due to their intrinsic exponential losses. However, it's very challenging to transmit faint quantum optical pulses between a satellite and the Earth because of unavoidable high channel losses. To overcome this issue, a classical channel is used along the quantum channel for the purpose of synchronization. In this channel, a positioning sequence is transmitted from the satellite to the stationary ground. To guarantee the timing jitter performance, the transmitted sequence must also avoid a long period of no pulses. This thesis designs Run length limited de Bruijn sequences which are not only the positioning sequence but are also able to avoid a number of consecutive $0$ bits. Such subjects are expected to have various applications and they also present some interesting theoretical questions in combinatorics, algorithms and coding theory. This thesis provides the first explicit formula for the maximal length of the run length limited de Bruijn sequences. Furthermore, using Lyndon words, an efficient construction of a run length limited de Bruijn sequence with the maximal length is presented. In addition, a sub-linear decoding algorithm which can locate the position of an arbitrary substring is also provided. 

% \begin{flushright}
% 	\begin{minipage}[t]{0.5\textwidth}
% 		\begin{center}
% 			\textit{Ha Noi}, xx May 2022\\[2cm]
		
% 			\textit{Student}
% 		\end{center}
% 	\end{minipage}
% \end{flushright}
% \pagebreak