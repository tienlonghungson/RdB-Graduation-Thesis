\begin{center}
    \Large{\textbf{ABSTRACT}}\\
\end{center}
\vspace{1cm}

Quantum key distribution (\gls{qkd}) is a secure communication enabling two parties to produce a shared random secret key known only to them. Current commercial deployed \gls{qkd} systems have transmission range restricted to under $1000$ km because they rely on optical fiber. The alternative method, satellite \gls{qkd}, is able to overcome this issue but faces a new challenge caused by noisy environments and swift relative motion between the transmitter and receiver. 

Therefore, a classical channel, which actually is a timing and synchronization system, is used along with the quantum channel. In such systems, Peide Zhang .et.al proposed to transmit a positioning sequence (also known as a de Bruijn sequence). To consider the timing jitter performance, a long period of no-pulses should be forbidden. In Peide Zhang's method, two pulse slots are used to represent a single bit (on-on is $1$ and on-off is $0$) so that one can avoid two consecutive no-pulses. However, the above scheme, called Hybrid de Bruijn (\gls{HdB}) code, requires $2n$ pulse slots to represent a de Bruijn sequence of length $n$ and it needs to receive a sub-sequence of $2 \log n$ pulse slots to locate its position. 

Observe that it is possible to use less redundant pulse slots to achieve both goals: to synchronize accurately and to avoid a long period of no-pulses, in this thesis, \gls{RdB} sequences are designed in which each binary bit is represented by only one pulse slot, $1$ is on and $0$ is off. The \gls{RdB} sequences are shown to be more general and efficient than the previous work. 

This thesis provides the first explicit formula for the maximal length of the run length limited de Bruijn sequences. Furthermore, using Lyndon words, an efficient construction of a run length limited de Bruijn sequence with the maximal length is presented. In addition, a sub-linear decoding algorithm that can locate the position of an arbitrary substring is also provided. 

% Satellite quantum key distribution is proposed to be an alternative method to establish intercontinental secure communication links instead of using optical fibre which is incapable of transmitting information in long distance due to their intrinsic exponential losses. However, it's very challenging to transmit faint quantum optical pulses between a satellite and the Earth because of unavoidable high channel losses. To overcome this issue, a classical channel is used along the quantum channel for the purpose of synchronization. In this channel, a positioning sequence is transmitted from the satellite to the stationary ground. To guarantee the timing jitter performance, the transmitted sequence must also avoid a long period of no pulses. This thesis designs Run length limited de Bruijn sequences which are not only the positioning sequence but are also able to avoid a number of consecutive $0$ bits. Such subjects are expected to have various applications and they also present some interesting theoretical questions in combinatorics, algorithms and coding theory. This thesis provides the first explicit formula for the maximal length of the run length limited de Bruijn sequences. Furthermore, using Lyndon words, an efficient construction of a run length limited de Bruijn sequence with the maximal length is presented. In addition, a sub-linear decoding algorithm which can locate the position of an arbitrary substring is also provided. 

% \begin{flushright}
% 	\begin{minipage}[t]{0.5\textwidth}
% 		\begin{center}
% 			\textit{Ha Noi}, xx May 2022\\[2cm]
		
% 			\textit{Student}
% 		\end{center}
% 	\end{minipage}
% \end{flushright}
% \pagebreak