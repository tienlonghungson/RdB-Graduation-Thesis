Such flaws of \citeauthor{zhang2021timing}'s system, fortunately, can still be improved. That is the goal this thesis aims to. The main task here is to design a code satisfying the constraints of \gls{dBTS} system.

\textbf{Problem Statement}: Designing a high rate sequence that is capable of positioning and avoids long periods with no pulse

This problem is similar to constructing a constrained positioning sequence. In this thesis, \gls{RdB} are developed. The longest \gls{RdB} sequences, not only have a high rate but also are generated and decoded rapidly. In summary, the contributions of this thesis are listed as follows:
\begin{itemize}
    \item Proposing a new kind of sequence (\gls{RdB}), more efficient (higher rate, more general and adaptive) than \gls{HdB} sequences.
    \item Determining the length of the longest \gls{RdB} sequences.
    \item Determining the maximal asymptotic rate of \gls{RdB} sequences.
    \item Providing fast encoder and decoder based on state-of-the-art algorithms.
\end{itemize}

The rest of this thesis is organized as follows. \textbf{Chapter 2} gives a brief introduction to coding theory and the application of de Bruijn sequences in such a research area. Other important results surrounding de Bruijn sequences and their generalizations are also provided. \textbf{Chapter 3} describes precisely the proposed run length limited de Bruijn sequence. For more understanding, the graph presentation of such sequences is presented. \textbf{Chapter 4} studies the properties of the longest run length limited de Bruijn sequence. This chapter answers the major questions: How long is that sequence? What is its rate and maximal asymptotic rate? How to generate the longest run length limited de Bruijn sequence of order $k$? Also, how to locate the position of each length $k$ sub-string of such a sequence?

% Chapter \ref{chapter:RdB} presents the proposed sequence of this thesis, which is called Run length limited de Bruijn sequence (\gls{RdB}). The \gls{RdB} sequence is not just suitable with \gls{dBTS} system, but also have a higher rate than \gls{HdB} sequence. More precisely, rate of the longest \gls{RdB} sequence is \[\log\left(\dfrac{1+\sqrt{5}}{2}\right)\approx0.6942.\]

% Chapter \ref{chapter:pro_rlldb_sequence} provides the efficient algorithm to generate one of the longest \gls{RdB} sequence. This algorithm, called an encoder, is based on the \gls{fkm}, which is the fastest method to produce a de Bruijn sequence. Moreover, to locate the position of an abitrary proper subsequence in the whole \gls{RdB} sequence, a decoder is also presented. The proposed decoder modifies the decoding algorithm found by Kociukima .et.al in \cite{kociumaka2016efficient}, which is currently the state of the art method to position a subsequence in the de Bruijn sequence, and therefore, is better than look-up table.

% Beside, the \gls{RdB} sequence is even more general and adaptive. More particularly, when the constraint of forbidding pattern $00$ is relaxed, that is, a longer run of bit $0$'s is allowed, the \gls{RdB} sequence can be easily adjusted to make the its rate higher and still suits the system. 

