The reason why de Bruijn graph, its sequence, and their generalizations are having so much attention is due to their diverse important applications. Very soon after the formal definition of this graph was given birth, one of its first applications was found in the introduction of shift-register sequences in general and linear feed-back registers in particular~\cite{golomb19821967}. Throught out the years, these type of sequences and graphs have found a variety of applications.

In cryptography, for example, the Baltimore Hilton Inn used de Bruijn sequence to install a cipher lock system for each of its rooms in lieu of the conventional key-lock system~\cite{fredricksen1982survey}. The low-cost n-stage shift register was used to generate maximum-length pseudorandom sequences in stream cipher, though later, this method was proved to be vulnerable to known-plaintext attack~\cite{lempel1979cryptology}.

De Bruijn sequences also opened a new field of research surround its complexity. Agnes Hui Chan et.al studied the complexity and the distribution of the complexities of de Bruijn sequences~\cite{chan1982complexities}. Especially, for binary sequence with period $2^n$, they come up with a fast algorithm determining its complexity~\cite{games1983fast}. Edwin on himself analysed the structure and complexity of nonlinear binary sequence generators~\cite{key1976analysis}. Tuvi et.al studied the error linear complexity spectrum of binary sequences with period $2n$~\cite{etzion2009properties}. Also Tuvi, in his joint work with Lampel~\cite{etzion1984construction}, found a construction of de Bruijn sequence to show that the lower bound of its complexity ($2^{n-1}+n$) is attainable for all $n$.

In~\cite{lempel1985design}, A.Lampel and M.Cohn are interested in designing an universal test sequeces for VLSI (very large scale integration chip). A binary sequence is called $(s,t)$-universal, $s>t$, if when shifted through a register of length $s$, it exercises every subset of $t$ register positions. Their proposed method was concatenating a set of de Bruijn sequences of appropriate length. In~\cite{barzilai1983exhaustive}, Zeev Barzilai .et.al also demonstrated an application of de Bruijn sequence in VLSI self-testing.

There are also other applications requiring two-dimensional version of de Bruijn sequeces. And the research about two-dimensional generalization of de Bruijn sequences comes to call. One well-known version is called pseudo-random arrays. In 1976, Mac Williams and Neil Sloane~\cite{macwilliams1976pseudo} gave a simple description of pseudo-random arrays and studied several of their nice properties. In 1988, Tuvi~\cite{etzion1988constructions}, represented a new version of pseudo-random arrays to construct perfect maps. Another approach by Bruck Stein~\cite{bruckstein2012simple}, he combined a de Bruijn sequence and a half de Bruijn sequece to study it robust and self-location properties. Studies~\cite{hsieh2001decoding,morano1998structured,pages2005optimised,salvi2010state,van1994digital} used pseudo-random arrays to with applications to robust undetectable digital watermarking of two-dimensional test images, and structured light. 

More surprisingly, de Bruijn's modern applications are even combined with biology, like the genome assembly as part of DNA sequencing. For example, Chaisson et.al~\cite{chaisson2009novo} described a new tool, EULER-USR, for assembling mate-paired short reads and use it to analyze the question of whether the read length matters. Compeau et.al~\cite{compeau2011apply} represented a method using de Bruijn graph to genome assembly. In 2001, Pevzner et.al~\cite{pevzner2001new} abandoned the classical “overlap - layout - consensus” approach in favor of a new Eulerian Superpath approach, that, for the first time, resolves the problem of repeats in fragment assembly. Later on, in 2003, Yu Zhang and Michael Waterman~\cite{zhang2003eulerian}, adapted the Pevzner's method to global multiple aligment for DNA sequences. In DNA storage, Han Mao et.al~\cite{chang2017rates,kiah2016codes} studied codes and their rates for DNA sequence profiles. Their studies were based on de Brujn graph.

In some new memory technologies, mainly in racetrack memories, and other ones which can be viewed as an $l$-read channel, synchronization errors (which are shift errors known also as deletions and sticky insertions) occur. By proposing a new de Bruijn based schema, used locally-constrained de Bruijn sequence to construct such code, Chee et.al~\cite{chee2021locally} are able to increase the rate of codes which correct the synchronization errors. Locally-constrained de Bruijn sequences and codes (sets of sequences) are of interest in their own right from both practical and theoretical points of view.

Recently, in 2021, a novel application of de Bruijn sequence has been found in quantum communication. Generally, to transmit quantum information between a satellite and the ground station, a timing and synchronization system has been used. Having observed that the intrinsic properties of positioning sequence are very suitable for this system, Peide Zhang et.al~\cite{zhang2021timing} have modulated it into \gls{HdB} sequence to transmit along the quantum channel. %More details about this application are provided and clarified in the following chapter, chapter~\ref{chapter:motivation}.

