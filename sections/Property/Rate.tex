For every $(k,s)$-RdB sequence, their rate is defined as follows:
\begin{definition}[Rate]
    Denote $R(\bfx_{k,s})$ to be the rate of a $(k,s)$-RdB sequence $\bfx_{k,s}$, then:
    \begin{align}
        R(\bfx_{k,s}) = \dfrac{\log\left(\card{\bfx_{k,s}}\right)}{k}
    \end{align}
    where the base of logarithm function is $\left\lvert\Sigma\right\rvert=2$.
\end{definition}

The rates of sequences are the proportion of the data stream that is useful (non-redundant), which tell how much useful information is transmitted. The sequence's rate actually originates from the information rate. Recall that, in definition~\ref{def:information_rate}, the rate $R_{C}$ of a code $C$ consisting of length $n$ $q$-ary sequences is $R_{C}=\dfrac{\log_{q}\left(\card{C}\right)}{n}$. If a $(k,s)$-RdB sequence $x_{k,s}$ is considered to be a code $C_{k,s}$, each of its size $k$ windows is treated as a codeword. The size of $C_{k,s}$ is exactly the length of $x_{k,s}$ minus $k$, but the offset $k$ can be omitted under log calculation. Hence: $$R_{C_{k,s}}= \dfrac{\log_{q}\left(\card{C_{k,s}}\right)}{n} = \dfrac{\log(\card{\bfx_{k,s}})}{k} = R(x_{k,s})$$. That is to say, $R(\bfx_{k,s})$ is eventually a kind of information rate.

Note that, a high rate is usually preferred. Accordingly, with each given pair $(k,s)$, the maximal rate of all $(k,s)$-RdB sequence is defined.
\begin{definition}[Maximal rate]
    Given $k$ and $s$, the maximal rate $R_{k,s}$ of all $(k,s)$-RdB sequences is:
    \begin{align}
        R_{k,s} = \dfrac{\log(N(k,s))}{k}
    \end{align}
\end{definition}

% For the rate $R_{k,s}$, if one cares about the very large $k$, it is when the maximal asymptotic rate of sequences comes to play.

The maximal asymptotic rate is concerned in case $k$ appears to be very large.
\begin{definition}[Maximal asymptotic rate]
    Denote $R_{s}$ to be the maximal asymptotic rate of $(k,s)$-RdB sequences, then:
    \begin{align}
        R_{s} = \lim_{k\to\infty}\dfrac{\log(N(k,s))}{k}
    \end{align}
\end{definition}

Having the explicit formula of $N(k,s)$ determined makes it easier to calculate the maximal asymptotic rate of the $(k,s)$-RdB sequence. The following equation is a direct consequence of theorem~\ref{theo:maximal_length}:
\begin{align}
    N(k,s) = \card{ W(k,s)} - \left(\sum_{i=0}^{C}\card{ W(k-i-s-3,s)} - s\right) + k - 1\label{eq:N_k_s}
\end{align}


Theorem \ref{theo:rate_1} below shows that the rate of $(k,1)$-RdB sequence is better than the rate of \gls{HdB} sequence. Therefore, it's able to use $(k,1)$-RdB sequence for the system in \cite{zhang2021timing} instead to increase the rate, speed of encoding, and decoding of the transmitted signals.

\begin{theorem}\label{theo:rate_1}
    \begin{align}
        R_{1} = 0.6942
    \end{align}
\end{theorem}
\begin{proof}
    Substitute $s=1$ into expression~\ref{eq:N_k_s} gives $N(k,1) = \card{W(k,1)} - \card{W(k-4,1)}$, and recall that $\{\card{W(k,1)}\}$ is a Fibonacci sequence with: 
    \[\card{W(k,1)} = \dfrac{\phi^{k+2}+\varphi^{k+2}}{\sqrt{5}}\].
    where $\phi = \dfrac{1+\sqrt{5}}{2}$ and $\varphi = \dfrac{1-\sqrt{5}}{2}$. Here, observe that $\phi\varphi=-1$, so $\phi^{2}\varphi^{2}=1$, consequently, it's able to write:
    \begin{align*}
        \left\{\begin{matrix}
            \phi^{-2} = \varphi^{2} \\
            \varphi^{-2} = \phi^{2}
        \end{matrix}\right.
    \end{align*}
    As a result: 
    \begin{align*}
        \card{W(k,1)} - \card{W(k-4,1)} &= \dfrac{\phi^{k+2}+\varphi^{k+2}}{\sqrt{5}} - \dfrac{\phi^{k-2}+\varphi^{k-2}}{\sqrt{5}}\\
        &= \dfrac{\phi^{k+2}+\varphi^{k+2}-\phi^{k}\varphi^{2}-\varphi^{k}\phi^{2}}{\sqrt{5}}\\
        &= \left(\phi^{k}+\varphi^{k}\right).\dfrac{\phi^{2}-\varphi^{2}}{\sqrt{5}}
    \end{align*}
    And hence:
    \begin{align*}
        R_{1} &= \lim_{k\to\infty}\dfrac{\log(N(k,1))}{k} \\
        &= \lim_{k\to\infty}\dfrac{ \log\left( \left(\phi^{k}+\varphi^{k}\right).\dfrac{\phi^{2}-\varphi^{2}}{\sqrt{5}}+k \right) }{k} \\
        &= \log(\phi) \approx 0.6942
    \end{align*}
\end{proof}

Note that $\phi$ is the largest root of equation $x^{2}-x-1=0$. Based on that observation, the prediction here is that if $\omega$ is the root of equation:
\begin{equation}
    x^{s+1}-x^{s}-\ldots-x-1 = 0\label{eq:char_eq}
\end{equation}
satisfying $\card{\omega}$ is the largest, the maximal asymptotic rate $R_{s}=\log(\lvert\omega\rvert)$. Fortunately, this is proved to be true in the following theorem, the generalization of theorem \ref{theo:rate_1}.
    

% The following theorem is the generalization of theorem \ref{theo:rate_1}.
\begin{theorem}[Maximal asymptotic rate of RdB sequence]\label{theo:rate_s}
    Let $s$ be a positive integer. Then:
    \begin{align}
        R_{s} = \log(\lvert\omega\rvert)
    \end{align}
    % where $\omega$ is the root of equation $a=b+1\inlineeqno$:
    % \begin{equation}
    %     x^{s+1}-x^{s}-\ldots-x-1 = 0\label{eq:char_eq}
    % \end{equation}
    % such that $\lvert\omega\rvert$ is the largest.
\end{theorem}
\begin{proof}
    
    The root $\omega$ is actually a Pisot number. More particularly, $\omega$ is the only positive roots of~\ref{eq:char_eq} lying in the interval $(1,2)$, the other roots are in the open disk $\left\{z\in\mathbb{C},\card{z}<1\right\}$. Also, note that $x^{s+1}-x^{s}-\ldots-x-1=0$ is the characteristic equation of $W(k,s)$. Therefore, with $k$ big enough, $\card{W(k,s)}$ can be estimated as follows: \[\card{W(k,s)} \approx a.\lvert\omega\rvert^{k}\] whereas $a$ is a positive constant. Thus, from expression~\ref{eq:N_k_s}:
    \begin{align*}
        N(k,s) &= \card{ W(k,s)} - \left(\sum_{i=0}^{C}\card{ W(k-i-s-3,s)} - s\right) + k - 1 \\
        &\approx \sum_{t=k-s}^{k}(k-t+1)a\lvert\omega\rvert^{t-2} + s + k - 1\\
        &= \lvert\omega\rvert^{k-s-2}\left( \sum_{t=k-s}^{k}a(k-t+1)\lvert\omega\rvert^{t-k+s} + \dfrac{s+k-1}{\lvert\omega\rvert^{k-s-2}}\right)\\
        &= \lvert\omega\rvert^{k-s-2}\left( \sum_{i=0}^{s}a(i+1)\lvert\omega\rvert^{s-1} + \dfrac{s+k-1}{\lvert\omega\rvert^{k-s-2}}\right) 
    \end{align*}

    This results in:
    \begin{align*}
        R_{k} &= \lim_{k\to\infty}\dfrac{\log\left( \lvert\omega\rvert^{k-s-2}\left(\sum_{i=0}^{s}a(i+1)\lvert\omega\rvert^{s-1}+\dfrac{s+k-1}{\lvert\omega\rvert^{k-s-2}}\right)\right)}{k} \\
        &= \lim_{k\to\infty}\dfrac{(k-s-2)\log(\lvert\omega\rvert)}{k} + \lim_{k\to\infty}\dfrac{\log\left(\sum_{i=0}^{s}a(i+1)\lvert\omega\rvert^{s-i}+\dfrac{s+k-1}{\lvert\omega\rvert^{k-s-2}}\right)}{k}\\
        &= \log(\lvert\omega\rvert)
    \end{align*}
    
\end{proof}

Table~\ref{tab:log_omega} lists the first $8$ values of $\log(\omega)$ with respect to $s$ from $1$ to $8$. 
\begin{table}[htbp]
    \centering
    \begin{tabular}{||m{0.95cm}||c|c|c|c|c|c|c|c|c|}
        \hline
        $s$ & 1 & 2 & 3 & 4 & 5 & 6 & 7 & 8 \\
        \hline
        $\log(\omega)$ & 0.6942 & 0.8791 & 0.9468 &	0.9752 & 0.9881 & 0.9942 & 0.9971 & 0.9986 \\
        \hline
    \end{tabular}
    \caption{Values of $\log(\omega)$ with $s$ from $1$ to $8$}
    \label{tab:log_omega}
\end{table}

To demonstrate the convergence of $R_{k,s}$ to $\log(\card{\omega})$, table~\ref{tab:convergence} lists the first values of $R_{k,s}$ with $k=\overline{2,4}$ and then increase to the large $k=\overline{71,74}$.
\begin{table}[htbp]
    \centering
    \begin{tabular}{||c||c|c|c|c|c|c|c|c|c}
        \hline
        \diagbox{s}{k} & 2 & 3 & 4 & \dots & 72 & 73 & 74 & $\log(\omega)$ \\
        \hline\hline
        1 & 1      & 0.9358 & 0.8649 & \dots & 0.6942 & 0.6942 & 0.6942 & 0.6942 \\ \hline
        2 & 1.1610 & 1.0566 & 1      & \dots & 0.8791 & 0.8791 & 0.8791 & 0.8791 \\ \hline
        3 & 1.1610 & 1.1073 & 1.0425 & \dots & 0.9468 & 0.9468 & 0.9468 & 0.9468 \\ \hline
        4 & 1.1610 & 1.1073 & 1.0620 & \dots & 0.9752 & 0.9752 & 0.9752 & 0.9752 \\ \hline
        5 & 1.1610 & 1.1073 & 1.0620 & \dots & 0.9881 & 0.9881 & 0.9881 & 0.9881 \\ \hline
        6 & 1.1610 & 1.1073 & 1.0620 & \dots & 0.9942 & 0.9942 & 0.9942 & 0.9942 \\ \hline
        7 & 1.1610 & 1.1073 & 1.0620 & \dots & 0.9971 & 0.9971 & 0.9971 & 0.9971 \\ \hline
        8 & 1.1610 & 1.1073 & 1.0620 & \dots & 0.9986 & 0.9986 & 0.9986 & 0.9986 \\ \hline
    \end{tabular}
    \caption{The convergence of $R_{k,s}$} with $s$ from $1$ to $8$
    \label{tab:convergence}
\end{table}
% \begin{figure*}[!ht]
%     \centering
%       \begin{subfigure}{0.9\linewidth}
%         \centering
%         \includegraphics[scale=0.23]{fig/Legend.png}
%         \label{fig:neg_size}
%         \end{subfigure}
%         %\quad
%       \begin{subfigure}{0.4\linewidth}
%         \centering
%         \includegraphics[scale=0.35]{fig/train_percent_Hit@1.pdf}
%         \caption{Hit@1}
%         \label{fig:sub_hit1}
%         \end{subfigure}
%         %\quad
%       \begin{subfigure}{0.4\linewidth}
%         \centering
%         \includegraphics[scale=0.35]{fig/train_percent_Hit@10.pdf}
%         \caption{Hit@10}
%         \label{fig:sub_hit10}
%         \end{subfigure}
%         %\quad
%     \begin{subfigure}{0.4\linewidth}
%         \centering
%         \includegraphics[scale=0.35]{fig/train_percent_MRR.pdf}
%         \caption{MRR}
%         \label{fig:sub_mrr}
%         \end{subfigure}
%     \caption{Saving of labelling effort for entity alignment on D-W-V1 test set}
%     \label{fig:supervision_percent}
%  \end{figure*}